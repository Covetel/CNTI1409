%% LyX 1.5.5 created this file.  For more info, see http://www.lyx.org/.
%% Do not edit unless you really know what you are doing.
\documentclass[spanish]{scrartcl}
\usepackage[T1]{fontenc}
\usepackage[latin9]{inputenc}
\usepackage{graphicx}

\makeatletter

%%%%%%%%%%%%%%%%%%%%%%%%%%%%%% LyX specific LaTeX commands.
%% A simple dot to overcome graphicx limitations
\newcommand{\lyxdot}{.}


\makeatother

\usepackage{babel}
\deactivatetilden

\begin{document}

\titlehead{\includegraphics{/home/elsanto/covetel/validador_cnti/branches/walter/ValidadorCNTI/root/static/images/logo_cnti.jpeg}}


\title{PROTOCOLO DE PRUEBA DE CARACTER�STICAS T�CNICAS DEL PORTAL {[}\%portal\%]\\
NT CNTI 0003-1: 2008 %
\thanks{Este informe fue preparado por el equipo de desarrollo de la Cooperativa
Venezolana de Tecnolog�as Libres Covetel R.S.%
}}

\maketitle
\tableofcontents{}


\section{Introducci�n}

Este documento tiene como objetivo describir el instrumento a trav�s
del cual se especifican, desde la perspectiva de la Gerencia de Normalizaci�n
y Certificaci�n enmarcado en el \textquotedblleft{}Proyecto de Certificaciones
y est�ndares en Tecnolog�as de Informaci�n para el Estado\textquotedblright{},
todos los aspectos necesarios que deben llevarse a cabo en el proceso
de evaluaci�n de Productos y Servicios de Tecnolog�as Libres de acuerdo
a lo esperado y basado en la disposiciones de la Norma T�cnica de
Caracter�sticas T�cnicas para Portales de Internet, C�digo NT CNTI
0003-1: 2008. La realizaci�n de las pruebas constituye una herramienta
fundamental para la verificaci�n adecuada de las disposiciones indicadas
en la norma. Este Protocolo establece un procedimiento de pruebas
en funci�n a los criterios establecidos en la norma t�cnica, adem�s
de emitir resultados, observaciones y acciones correctivas de dichas
pruebas.


\section{Objetivo}

Establecer un procedimiento de pruebas que sirva de instrumento en
el proceso de evaluaci�n de Productos y Servicios de Tecnolog�as de
Informaci�n Libres, que permita verificar el cumplimiento de las disposiciones
de la Norma T�cnica para Portales de Internet, referidas a su Nombre
de Dominio, Versi�n de HTML,Consideraciones de Navegabilidad, Diagramaci�n
Gr�fica, Juego de Caracteres, Lenguaje Script, Controladores A�adibles,
Hojas de estilo en cascada, Meta Etiquetas y Pol�ticas de Seguridad. 


\section{Alcance}

En el presente documento Protocolo de Prueba se plantea evaluar el
grado de satisfacci�n en distintos aspectos funcionales y no funcionales
basados en la disposiciones de la norma Caracter�sticas T�cnicas de
Portales de Internet. 


\section{Resultados}


\subsection{Resultados Generales.}

{[}\% cumple \%]


\subsection{Disposiciones Evaluadas. }


\subsubsection{Nombre del Dominio}

Todo Portal de Internet de los �rganos y entes de la Administraci�n
P�blica Nacional deber� ser accesible a trav�s de un nombre de dominio
compuesto por el dominio de segundo nivel que represente el nombre
del �rgano o ente encargado del Portal de Internet, un dominio gen�rico
de segundo nivel (.gob), seguido del dominio de primer nivel de pa�s
(.ve). 

\textbf{Resultado:} {[}\% dominio \%]


\subsubsection{Versi�n de HTML }

Los documentos de hipertexto a ser usados en los Portales de Internet
deber�n utilizar como formato las especificaciones HTML 4.01, seg�n
se expresa en las recomendaciones HTML 4.01 del 24 de diciembre de
1999, o XHTML 1.0 de fecha 1 de agosto de 2002, de la W3C . Los documentos
de hipertexto de los Portales de Internet sujetos a esta Norma deben
ser validados utilizando las herramientas que la W3C dispone en l�nea,
procur�ndose la inclusi�n gr�fica en los Portales de los sellos de
conformidad dispuestos por la W3C. 

\textbf{Resultado:} {[}\% dominio \%]
\end{document}
